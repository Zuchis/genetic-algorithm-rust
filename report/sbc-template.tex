\documentclass[12pt]{article}
\usepackage{sbc-template}
\usepackage{enumerate}
\usepackage{hyperref} 
\usepackage{url} 
\usepackage{tikz}
\usepackage[utf8]{inputenc}
\usepackage[brazil]{babel}
\usepackage[T1]{fontenc}
\usepackage[scaled=0.85]{beramono}
\usepackage{csquotes}
\usepackage{listings}
\usepackage{color}
\usepackage{amsmath}
\usepackage{graphicx}
\renewcommand\lstlistingname{C\'odigo}
\setlength\parindent{20pt}

\usetikzlibrary{decorations.pathreplacing,calc}
\newcommand{\tikzmark}[1]{\tikz[overlay,remember picture] \node (#1) {};}

\newcommand\numberthis{\addtocounter{equation}{1}\tag{\theequation}}

\newcommand*{\AddNote}[4]{%
    \begin{tikzpicture}[overlay, remember picture]
        \draw [decoration={brace,amplitude=0.4em},decorate,black]
            ($(#3)!([yshift=1.5ex]#1)!($(#3)-(0,1)$)$) --  
            ($(#3)!(#2)!($(#3)-(0,1)$)$)
                node [align=center, text width=2.5cm, pos=0.5, anchor=west] {#4};
    \end{tikzpicture}
}%
     
\sloppy

%titulo
\title{Relatório Técnico Sobre uma Ferramenta para Algoritmos Genéticos}
%autor
\author{Vinicius Bruch Zuchi\inst{1} }

\address{Departamento de Ciência da Computação -- Universidade do Estado de Santa Catarina\\
  Centro de Ciências Tecnológicas -- Caixa Postal 15.064 -- Joinville -- SC -- Brasil
  \email{vinicius.b.zuchi@gmail.com}
}

\renewcommand\lstlistingname{C\'odigo}

\begin{document} 
\maketitle

\section{Introdução}

Algoritmos genéticos (AGs) são uma classe de algoritmos meta-heurísticos cujo objetivo é melhorar uma solução para um problema até que essa 
atinja, idealmente, o ótimo e caso não seja possível, o mais próximo possível.

Os algoritmos genéticos são inspirados pela teoria da seleção natural de Charles Darwin e para alcançar o seu objetivo são utilizados 
conceitos oriundos da teoria da seleção natural, como uma população de indivíduos, adaptabilidade e evolução natural, assim como conceitos 
da biologia, como gene, cromossomos, alelos, genótipo e fenótipo. A evolução natural nos algoritmos genéticos é simulada através do que é 
conhecido como operadores genéticos, tais como: seleção dos melhores indivíduos, cruzamento entre indivíduos e mutação genética.

Como a evolução ocorre sobre uma população, os algoritmos genéticos também precisam operar sobre uma população, ou seja, várias soluções 
diferentes, ao invés de somente uma. Para calcular o quão boa cada uma das soluções é, utiliza-se uma função conhecida como função de 
adaptabilidade (do inglês \textit{fitness function}), uma das partes mais importantes de se modelar um algoritmo genético é escolher uma 
função de \textit{fitness} condizente com a natureza do problema que está sendo resolvido. Além disso, também é de substancial importância 
escolher a codificação correta dos indivíduos, isto é, estruturalmente, como estes serão representados.

Um algoritmo genético possui uma considerável quantidade de parâmetros disponíveis e combinações disponíveis e também precisa repetir 
as mesmas funções diversas vezes pois opera sobre uma população de indivíduos, e não somente sobre um. Por estes motivos, é necessário 
que a ferramenta criada para executar AGs seja modular, eficiente e otimizada, para alcançar tais propósitos, a linguagem de programação 
escolhida para a ferramenta foi a linguagem Rust.

Rust é uma linguagem de programação de baixo nível, com tipagem estática e forte. A linguagem foi projetada com os objetivos 
de ser rápida, fácil de ser paralelizada e ter segurança de memória, prevenindo a ocorrência de condições de corrida, estouros 
de pilha, e acessos a posições de memória não inicializadas ou desalocadas~\cite{Matsakis:2014:RL:2692956.2663188}. 

\section{Parametrização da Ferramenta}

Para a parametrização da ferramenta, foi criado um arquivo de configuração, no qual cada linha possui o nome de um parâmetro, seguido de 
seu respectivo valor, esse arquivo é então interpretado na ferramenta em tempo de execução, atribuindo cada parâmetro do arquivo de 
configuração para seu equivalente em uma estrutura que representa a população.

Os parâmetros disponíveis estão listados abaixo: 
\begin{itemize}
    \item tipo do gene: pode ser inteiro, real ou binário.
    \item tamanho da população.
    \item tamanho de cada indivíduo.
    \item limite inferior e superior para valores do tipo inteiro ou real.
    \item probabilidade de cruzamento e mutação.
    \item número de iterações.
    \item método de seleção.
    \item tamanho do torneio, caso o método de seleção seja torneio.
    \item método de cruzamento.
    \item função de adaptabilidade.
    \item método de mutação.
    \item elitismo booleano para ativar ou desativar o elitismo.
\end{itemize}

Internamente no programa, há uma estrutura que representa uma população, e nesta há um atributo para cada um desses parâmetros, para o caso 
dos parâmetros que são métodos, é armazenado na estrutura ponteiros para funções existentes em seus respectivos módulos.

Para cada parte do algoritmo genético como: cruzamento, \textit{fitness}, mutação, população e seleção, há um módulo correspondente que 
implementa todas as funções relacionadas a estas partes, isso permite uma melhor organização da ferramenta e facilita a adição de novas funcionalidades.

\section{Seleção}

Essa seção discute o conceito e implementação dos métodos de seleção implementados, todos os métodos de seleção recebem como argumento uma população e 
retornam o índice do indivíduo escolhido.

\subsection{Roleta}

Este método possui uma certa probabilidade para selecionar qualquer indivíduo da população, todos possuem uma chance de serem escolhidos, 
mesmo que seja praticamente impossível. Este método inicialmente realiza um somatório sobre todos os \textit{fitnesses}, procede então para a 
geração de um número aleatório entre zero e o resultado deste somatório. O algoritmo então irá sequencialmente somar o \textit{fitness} de cada 
indivíduo em uma variável acumuladora que se inicia em zero. Para cada soma feita, é verificado se o acumulador passou o valor do número gerado 
aleatoriamente, e caso positivo, então esse será o indivíduo escolhido, desta maneira, indivíduos com um \textit{fitness} maior terá mais chances 
de ser escolhido, pois a probabilidade do valor somado passar o número aleatório é maior.

\subsection{Torneio}

Este método seleciona k indivíduos aleatoriamente, sendo k o número escolhido para ser o tamanho do torneio, e para todos os indivíduos selecionados 
verifica-se qual deles possui o maior \textit{fitness}, aquele que possuir o maior, será o escolhido pelo algoritmo. Este método não considera todos os 
indivíduos pois o número escolhido para o torneio não deve ser o mesmo do que o tamanho da população, e quanto maior o número de participantes do torneio, 
mais elitista a seleção se tornará.
    
\bibliographystyle{sbc}
\bibliography{sbc-template}

\end{document}
